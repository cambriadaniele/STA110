\chapter{Mock exam 1}%
\label{cha:Mock exam 1}
Solve with the pdf of the mock exam on the side.

\textbf{Notation:} We recall some of the terminology:
\begin{itemize}
    \item Given a nonempty set $\Omega$, $\mathcal{P}(\Omega)$ is the power set on $\Omega$;
    \item $\mathfrak{B}(\mathbb{R}^k)$ denotes the Borel $\sigma$-field on $\mathbb{R}^k$, $k \geq 1$;
    \item The measure
    \[
    \mu(A) = 
    \begin{cases} 
    \#A, & \text{if } A \text{ is finite} \\
    \infty, & \text{otherwise},
    \end{cases}
    \quad A \in \mathcal{P}(\Omega),
    \]
    is referred to as the counting measure on $\mathcal{P}(\Omega)$;
    \item Given a measurable space $(\Omega, \mathcal{F})$ and $x \in \Omega$, we write $\delta_x$ for the measure
    \[
    \mathcal{F} \ni A \mapsto \delta_x(A) = 
    \begin{cases} 
    1, & \text{if } x \in A \\
    0, & \text{otherwise}.
    \end{cases}
    \]
\end{itemize}

\begin{exercise}[]
    \label{ex:mock1}
    \begin{enumerate}[label=(\alph*)] \hfill
        \item Refer to Def. ~\ref{def:4.1}.

        \item Measure on $\mathcal{F}$ (cf. Def ~\ref{def:5.1}).
            \begin{itemize}
                \item 
                    \begin{enumerate}[label=(\roman*)]
                        \item $\mu_1(\emptyset) = C\mu(\emptyset)= 0$;
                        \item We know that item ii holds for the counting measure by definition. For our redefined
                            counting measure,
                            \[
                            \mu_1(\bigcup_{i \in \mathbb{N}} A_i) = C\mu(\bigcup_{i \in \mathbb{N}} A_i)
                            = C \sum_{i \in \mathbb{N}}^{ } \mu(A_i) = \sum_{i \in \mathbb{N}}^{ } C\mu(A_i)
                            = \sum_{i \in \mathbb{N}}^{ } \mu_1(A_i)
                            .\] 
                    \end{enumerate}

                \item
                    \begin{enumerate}[label=(\roman*)]
                        \item $\mu_2(\emptyset) = \int_{\emptyset} f(\omega)\mu(d\omega) = 0$;
                        \item 
                            \[
                            \mu_2(\bigcup_{i \in \mathbb{N}} A_i) = \int_{\bigcup_{i \in \mathbb{N}} A_i} f(\omega)\mu(d\omega)
                            \stackrel{\text{Tool ~\ref{tool:8.9}}}
                            {=} \sum_{i \in \mathbb{N}}^{ }\int_{A_i}  f(\omega)\mu(d\omega) 
                            = \sum_{i \in N}^{ } \mu_2(A_i)
                            .\] 
                    \end{enumerate}

                \item
                    \begin{enumerate}[label=(\roman*)]
                        \item $\mu_3(\emptyset) = \frac{1}{2} + \lambda(\emptyset) = \frac{1}{2}$. 
                    \end{enumerate}
            \end{itemize}
            We see that  $\mu_3$ is clearly not a measure on $\mathcal{F}$.

        \item Probability measure cf. Def. ~\ref{def:10.1}.
            \begin{itemize}
                \item 
                    \[
                    P_1(\mathbb{R}) = \int_{\mathbb{R}} \mathbbm{1}_{[0, \infty)}(x) e^{-x}dx
                    = \int_{[0, \infty)} e^{-x}dx 
                    = (-e^{-x})|_0^{\infty} = (0 - (-1)) = 1
                    .\] 
                \item
                    \[
                    P_2(\mathbb{N}) = \int_{\mathbb{N}} \mathbbm{1}_{\{0,1\} }(x) x^{2}\mu(dx) 
                    = \int_{\{0,1\} } x^{2}\mu(dx) = 0^{2}\cdot \mu(\{0\} ) + 1^{2}\cdot \mu(\{1\} ) 
                    = 0\cdot 1 + 1 \cdot 1 = 1
                    .\] 
                \item
                    \begin{tool}[Integral with respect to a dirac measure]
                        \label{tool:iwrtadm} 
                        \[
                        \int_{\Omega} f(x) \delta_\omega(dx) = f(\omega)
                        .\] 
                    \end{tool}
                    \[
                    P_3(\mathbb{R}) = \int_{\mathbb{R}} x^{2}\mu(dx) 
                    = \int_{\mathbb{R}} x^{2} (\delta_{-1}(dx)+ \delta_1(dx))
                    = \int_{\mathbb{R}} x^{2} \delta_{-1}(dx) + \int_{\mathbb{R}}x^{2} \delta_1(dx)
                    \] 
                    \[
                    = (-1)^{2} + 1^{2} = 2
                    .\] 
            \end{itemize}
            We see that $P_3$ is not a probability measure on $\mathcal{B}$.

        \item Calculate:
            \begin{enumerate}[label=\arabic*.]
                \item $\lambda$ Lebesgue measure on $\mathfrak{B}(\mathbb{R})$.
                    \[
                    \int_{\mathbb{R}} \mathbbm{1}_{[-1,1]}(x) \lambda(dx)
                    = \int_{[-1,1]} 1 \lambda(dx) = 1 \cdot \lambda([-1,1]) = 1 \cdot 2 = 2
                    .\] 
                \item $P(A) = (1-p)\delta_0(A) + p \delta_1(A)$, $A \in \mathfrak{B}(\mathbb{R})$, $p \in (0,1)$.
                    \[
                    \int_{\mathbb{R}} (x-p)^{2}P(dx)
                    \] 
            \end{enumerate}
    \end{enumerate}
\end{exercise}
