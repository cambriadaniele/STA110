\chapter{Collection of random vectors}%
\label{cha:Collection of random vectors}

\section{Independence}%
\label{sec:Independence}

\begin{definition}[Independent sub-$\sigma$-fields]
    \label{def:11.1}
    Let $(\Omega, \mathcal{F}, \mathbb{P})$ be a probability space and $\mathcal{A}_1, \ldots,
    \mathcal{A}_n$ be $n$ sub-$\sigma$-fields on $\Omega$. $\mathcal{A}_1, \ldots, \mathcal{A}_n$
    are said to be independent if for any $A_1 \in \mathcal{A}_1, \ldots, A_n \in \mathcal{A}_n$,
    \[
    \mathbb{P}(A_1 \cap \ldots \cap A_n) = \mathbb{P}(A_1)\times \ldots \times \mathbb{P}(A_n)
    .\] 
\end{definition}

\begin{definition}[]
    \label{def:11.2}
    TODO: Understand if it's useful. If so, write it down and explain it.
\end{definition}

\begin{example}[]
    \label{ex:11.1}
    Quite clear. it is implied that after that the ball is drawn, it has to be put back into
    the urn.
\end{example}

\section{Sums of independent random vectors}%
\label{sec:Sums of independent random vectors}

\section{Gauss vectors}%
\label{sec:Gauss vectors}
\begin{definition}[Gauss vector]
    \label{def:11.6}
    A random vector $X=(X_1, \ldots, X_k)$ is said to be a Gauss vector if and only if for any 
    $v \in \mathbb{R}^{k}$, the random variable
    \[
    v^{t}X = v_1X_1 + \ldots + v_kX_k,
    \] 
    is Gaussian.
\end{definition}

\begin{remark}[]
    \label{rem:11.11}
    TODO
\end{remark}

