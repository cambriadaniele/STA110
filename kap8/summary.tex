\chapter{Integration: Part I}%
\label{cha:Integration: Part I}
\section{The integral for non-negative functions}%
\label{sec:The integral for non-negative functions}

If $f: \Omega \to \overline{\mathbb{R}}$ is s.t. $f(\omega)\ge 0$ for any $\omega \in \Omega$,
$f$ is said to be nonnegative.

\begin{definition}[Finite partitions]
    \label{def:8.1}
    Let $\Omega$ be a set. A partiton of $\Omega$ is a disjoint collection
    $\{A: A \in P\}, P \subset \mathcal{P}(\Omega)$, s.t. $\cup_{A \in P}A = \Omega$. That is,
    a partition of $\Omega$ is a disjoint collection of subets of $\Omega$ whose union is
    $\Omega$. If $\xi$ is a partition of $\Omega$, a set $A \in \xi$ is referred to as an atom
    of $\xi$. A partition $\xi \text{ of } \Omega$ is said to be finite, if it contains a
    finite number of atoms.
\end{definition}

\begin{example}[Finite partition]
    \label{ex:8.1}
    Let $\Omega = \{0, 1, \ldots, N\} , N \in \mathbb{N}$. Then, $\xi = \{\{\omega\}:w \in 
    \Omega\} $ is a finite partition of $\Omega$. (Partition contains $N+1$ elements).
\end{example}

\begin{definition}[$Z_0^{\mathcal{F}}$]
    \label{def:8.2}
    Let $(\Omega, \mathcal{F})$ be a measurable space. We use the notation 
    $Z_0^{\mathcal{F}}(\Omega) =Z_0^{\mathcal{F}}$ for the set which contains all the finite
    partitions of $\Omega$ with atoms from $\mathcal{F}$. That is,
    \[
    Z_0^{\mathcal{F}} = \{\xi : \xi \text{ is finite partition of } \Omega\text{ s.t. for any A } 
    \in \xi, A \in \mathcal{F}\} 
    .\] 
\end{definition}

\begin{definition}[Integral for a nonnegative standard simple function]
    \label{def:8.3}
    Let $(\Omega, \mathcal{F}, \mu)$ be a measure space and $f:\Omega \to \overline{\mathbb{R}}$ 
    be nonnegative and $\mathcal{F}$ measurable. Then, we define
    \[
    S_\mu^{f}(\xi) = \sum_{A \in \xi} (\inf_{\omega \in A}f(\omega))\mu(A), \quad \xi
    \in Z_0^{\mathcal{F}},
    \] 
    Essentially, $S_\mu^f(\xi)$ approximates the integral of $f$ by considering the smallest 
    value $f$ takes on each piece of the partition and multiplying this by the measure of 
    the piece. And
    \[
    \int_{\Omega} f(\omega)\mu(d\omega) = \sup_{\xi \in Z_0^{\mathcal{F}}} 
    S_\mu^{f}(\xi)
    .\] 
    The integral of $f$ over $\Omega$ with respect to $\mu$, is the supremum of 
    $S_\mu^f(\xi)$ over all possible partitions $\xi$ of $\Omega$ in $Z_0^{\mathcal{F}}$.
    This definition captures the idea of the integral as the limit of finer and finer 
    approximations of $f$ by simple functions.
    Upon the latter definition, we deduce the integral for a (nonnegative) standard simple
    function (cf. Definition ~\ref{def:7.4}).
\end{definition}

\begin{proposition}[]
    \label{prop:8.1}
    TODO
\end{proposition}

\begin{myexample}[Integral of a nonnegative standard simple function]
    \label{myex:8.1}
    Let $(\Omega, \mathcal{F}, \mu)$ be a measure space with $\Omega = \{a, b, c, d\} $,
    $\mathcal{F} = \mathcal{P}(\Omega)$, and $\mu$ is the counting measure, i.e., $\mu(A)$ is
    the number of elements in $A$. Let $f: \Omega \to \overline{\mathbb{R}}$,
    \[
    f(\omega) = 
    \begin{cases}
        1 & \text{ if } \omega = a,\\
        2 & \text{ if } \omega = b,\\
        3 & \text{ if } \omega = c,\\
        0 & \text{ if } \omega = d
    \end{cases}
    \] 
    Consider the partition $\xi = \{\{a\} , \{b\} , \{c\} , \{d\} \} $. 
    $\inf_{\omega \in \{a\} }f(\omega) = 1$, $\inf_{\omega \in \{b\} }f(\omega) = 2$,
    $\inf_{\omega \in \{c\} }f(\omega) = 3$, $\inf_{\omega \in \{d\} }f(\omega) = 4$.
    Since each singleton set in $\xi$ as measure of $1$ under $\mu$,
    \[
    S_\mu^{f}(\xi) = (1 \times 1) + (2 \times 1) + (3\times 1) + (0\times 1) = 6
    \]
    if $\sup_{\xi \in Z_0^{\mathcal{F}}} S_\mu^{f} = 6$, which I think it should be, then
    $\int_{\Omega} f(\omega)\mu(d\omega)$ = 6.
\end{myexample}

\begin{example}[]
    \label{ex:8.2}
    Example 8.2 interesting and clear, TODO.
\end{example}

\begin{proposition}[Monotone convergence theorem]
    \label{prop:8.2}
    Let $(\Omega,\mathcal{F}, \mu)$ be a measure space and $f_n:\Omega \to \overline{\mathbb{R}}$,
    $ n \in \mathbb{N}$, be a sequence of nonnegative $\mathcal{F}$ measurable functions s.t.
    for any $\omega \in \Omega, f_n(\omega)\uparrow f(\omega)$ for some $f: \Omega \to 
    \overline{\mathbb{R}}$. Then,
    \[
    \int_{\Omega}f_n(\omega)\mu(d\omega)\uparrow \int_{\Omega}f(\omega)\mu(d\omega)   
    .\] 
\end{proposition}

\begin{proposition}[The integral of nonnegative functions is linear]
    \label{prop:8.3}
    Let $(\Omega, \mathcal{F}, \mu)$ be a measurable space, $f, g: \Omega \to \overline{\mathbb{R}}$
    be two nonnegative and $\mathcal{F}$ measurable functions. Given $\alpha, \beta \in  [0, \infty)$ 
    we have that
    \[
    \int_{\Omega} (\alpha f + \beta g)(\omega)\mu(d\omega) =
    \alpha \int_{\Omega} f(\omega)\mu(d\omega) + \beta \int_{\Omega} g(\omega)\mu(d\omega)
    .\] 
\end{proposition}

As a consequence of the latter two proposition we have the following result:

\begin{proposition}[]
    \label{prop:8.4}
    Let $(\Omega, \mathcal{F}, \mu)$ be a measure space and $f_i: \Omega \to
    \overline{\mathbb{R}}, i \in  \mathbb{N}$, be a sequence of nonnegative $\mathcal{F}$ 
    measurable functions, then
    \[
        \int_{\Omega}\left( \sum_{i\in \mathbb{N}} f_i \right) (\omega)\mu(d\omega) =
        \sum_{i \in \mathbb{N}} \left( \int_{\Omega} f_i(\omega)\mu(d\omega) \right) 
    .\] 
\end{proposition}

\begin{definition}[True almost everywhere $(a.e.)$]
    \label{def:8.4}
    Let $(\Omega, \mathcal{F}, \mu)$ be a measure space. Suppose that for any $\omega
    \in \Omega, S(\omega)$ is a statment on $\Omega$. We say $S$ is true $\mu$ almost
    everywhere $(a.e.)$ if $\mu(\{\omega: S(\omega) \text{ is false}\} ) = 0$.
\end{definition}

\begin{example}[$\mu (a.e.)$]
    \label{ex:8.3}
    Interesting and clear. TODO.
\end{example}

\begin{proposition}[]
    \label{prop:8.5}
    Let $(\Omega, \mathcal{F}, \mu)$ be a measure space. Assume that $f, g: \Omega
    \to \overline{\mathbb{R}}$ be two nonnegatibe and $\mathcal{F}$ measurable functions.
    \begin{enumerate}[label=(\roman*)]
        \item If $\mu(\{\omega: f(\omega) > 0\} ) > 0, \text{ then } \int_{\Omega} f(\omega)
            \mu(d\omega)>0$;
        \item If $\int_{\Omega} f(\omega)\mu(d\omega) < \infty$, then $f<\infty \ \mu \ a.e.;$
        \item If  $f\le g \ \mu \ a.e.$, then $\int_{\Omega} f(\omega)\mu(d\omega)\le 
            \int_{\Omega} g(\omega)\mu(d\omega)$;
        \item If  $f = g \ \mu \ a.e.$, then $\int_{\Omega} f(\omega)\mu(d\omega) =
            \int_{\Omega} g(\omega)\mu(d\omega)$.
    \end{enumerate}
\end{proposition}

\section{Integrable functions}%
\label{sec:Integrable functions}
We recall the definiton of the positive $\left( f^{+} \right) $ and negative $\left( f^{-} \right) $
parts of a function (cf. Definition ~\ref{def:7.6}). Pay attention, $f^{-}$ is basically the negative
part of the function, but reflected by the x-axis. The result is positive. Also see ~\ref{ex:7.5}

\begin{definition}[Integral of an integrable function]
    \label{def:8.5}
    Let $(\Omega, \mathcal{F}, \mu)$ be a measure space  and $f: \Omega\to \overline{\mathbb{R}}$
    be a $\mathcal{F}$ measurable function. The integral of $f$ is defined by:
    \[
    \int_{\Omega} f(\omega)\mu(d\omega) = \int_{\Omega} f^{+}(\omega)\mu(d\omega) -
    \int_{\Omega} f^{-}(\omega)\mu(d\omega)
    ,\] 
    unless $\int_{\Omega} f^{+}(\omega)\mu(d\omega) = \int_{\Omega} f^{-}(\omega)\mu(d\omega) = 
    \infty$,
    in which case $\int_{\Omega} f(\omega)\mu(d\omega)$ is not defined. \\ If both 
    $\int_{\Omega} f^{+}(\omega)\mu(d\omega) 
    < \infty$ and $\int_{\Omega}f^{-}(\omega)\mu(d\omega) 
    < \infty$,
    $f$ is said to be integrable.

    (NOTE) This assumption is definied upon the measure $\mu$, i.e., if one wants to further refer to the measure
    of integration one specifies that $f$ is integrable with respect to $\mu$.
\end{definition}

\begin{proposition}[Generalisation of the condition for $f$ to be integrable]
    \label{prop:8.6}
    Let $(\Omega, \mathcal{F}, \mu)$ be a measure space  and $f: \Omega\to \overline{\mathbb{R}}$ be 
    $\mathcal{F}$ measurable. Then, $f$ is integrable if and only if
    $\int_{\Omega} \lvert f(\omega) \rvert \mu(d\omega) < \infty$.
\end{proposition}

\begin{proposition}[Extension (cf. (iii) Proposition ~\ref{prop:8.5})]
    \label{prop:8.7}
    Let $(\Omega, \mathcal{F}, \mu)$ be a measure space and $f, g: \Omega \to \overline{\mathbb{R}}$ 
    be $\mathcal{F}$ measurable. If $f$ and $g$ are integrable and $f\le g \ a.e.$, then,
    $\int_{\Omega} f(\omega)\mu(d\omega) \le \int_{\Omega}g(\omega)\mu(d\omega)$.
\end{proposition}

\begin{proposition}[Extension (c.f. Proposition ~\ref{prop:8.3})]
    \label{prop:8.8}
    Let $(\Omega, \mathcal{F}, \mu)$ be a measurable space, $f, g: \Omega \to \overline{\mathbb{R}}$
    be two integrable and $\mathcal{F}$ measurable functions. Then, for any $\alpha, \beta \in \mathbb{R}$ 
    we have that $\alpha f + \beta g$ is integrable and
    \[
    \int_{\Omega} (\alpha f + \beta g)(\omega)\mu(d\omega) =
    \alpha \int_{\Omega} f(\omega)\mu(d\omega) + \beta \int_{\Omega} g(\omega)\mu(d\omega)
    .\] 
\end{proposition}

\section{Fatou's lemma and Lebesgue's dominated convergence theorem}%
\label{sec:Fatou's lemma and Lebesgue's dominated convergence theorem}

\begin{proposition}[Fatou's lemma]
    \label{prop:8.9}
    Let $(\Omega, \mathcal{F}, \mu)$ be a measure space and $f_n: \Omega \to \overline{\mathbb{R}},
    n \in \mathbb{N}$, be a sequence of nonnegative and $\mathcal{F}$ measurable function. Then,
    \[
    \int_{\Omega} \lim_{n \to \infty} \inf f_n(\omega)\mu(d\omega) \le 
    \lim_{n \to \infty} \inf \int_{\Omega} f_n(\omega)\mu(d\omega)
    .\] 
\end{proposition}



