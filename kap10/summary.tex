\chapter{General notions in Probability}%
\label{cha:General notions in Probability}

\section{Probability spaces}%
\label{sec:Probability spaces}

\begin{definition}[]
    \label{def:10.1}
    Let $(\Omega, \mathcal{F})$ be a measurable space. A probability $\mathbb{P}$ on
    $\mathcal{F}$ is a measure on $\mathcal{F}$ s.t. $\mathbb{P}(\Omega) = 1$.
    The triple $(\Omega, \mathcal{F}, \mathbb{P})$ is referred to as a probability space.
\end{definition}

\begin{example}[]
    \label{ex:10.1}
    Let $\Omega$ be a finite and nonempety set. Define
    \[
    \mathbb{P}(A) = \frac{\#A}{\#\Omega}, \quad A \in \mathcal{P}(\Omega),
    .\] 
    Where $\mathcal{P}(\Omega)$ is the power set on $\Omega$. Then, $\mathbb{P}$ is a 
    probability on $\mathcal{P}(\Omega)$.
\end{example}

\begin{example}[]
    \label{ex:10.2}
    Let $C$ be a set s.t. $\#C = 52$. Suppose that
    \[
    C = S_1 \cup S_2 \cup  S_3 \cup S_4
    ,\] 
    with $\{S_1, S_2, S_3, S_4\} $ disjoint and s.t. $\#S_i = 13 \text{ for all } 
    i = 1,2,3,4$. We remain in the setting of the previous example with
    \[
    \Omega = \{A \subset C: \#A = 5\},
    \] 
    and $\mathbb{P}$ on $\mathcal{P}(\Omega)$ defined as in exercise ~\ref{ex:10.1}.
    Upon exercise ~\ref{ex:1.11}, we already know that $\#\Omega = \binom{52}{5}$.
    Let
    \[
    A_i = \{A \subset S_i: \#A = 5\}, \quad i=1,2,3,4, 
    \] 
    TODO
\end{example}

\section{Random variables and random vectors}%
\label{sec:Random variables and random vectors}

\begin{definition}[Random variable]
    \label{def:10.2}
    Let $(\Omega, \mathcal{F})$ be a measurable space. A map $X: \Omega \to \mathbb{R}$
    is referred to as a random variable on $(\Omega, \mathcal{F})$ if it if 
    $\mathcal{F}/ \mathfrak{B}(\mathbb{R})$ measurable.
\end{definition}

\begin{definition}[Random vector]
    \label{def:10.3}
    Let $(\Omega, \mathcal{F})$ be a measurable space. A map $X: \Omega \to \mathbb{R}^{k}$ 
    is referred to as a random vector on $(\Omega, \mathcal{F})$ if it is
    $\mathcal{F}/\mathfrak{B}(\mathbb{R})$ measurable.
\end{definition}

\begin{proposition}[]
    \label{prop:10.1}
    Let $(\Omega, \mathcal{F}, \mathbb{P})$ be a probability space and $X$ be a random
    vector on $(\Omega^{ }, \mathcal{F}^{})$. A random variable $Y$ on $(\Omega^{ }, \mathcal{F}^{ })$
    is $\sigma(X)$ measurable if and only if there exists a function
    $f: \mathbb{R}^{k} \to \mathbb{R}$ which is $\mathfrak{B}(\mathbb{R}^{k})$ measurable s.t.
    $Y = f(X)$.
\end{proposition}

\begin{definition}[]
    \label{def:10.4}
    Let $(\Omega, \mathcal{F}, \mathbb{P})$ be a probability space. The distribution or law of
    a random vector on $(\Omega^{ }, \mathcal{F}^{ })$ is the pushforward measure
    $P_X = \mathbb{P} X^{-1}$ on $\mathfrak{B}(\mathbb{R}^{k})$ (cf. Defintion ~\ref{def:9.1}).

\end{definition}





