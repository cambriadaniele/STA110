\chapter{Integration: Part II}%
\label{cha:Integration: Part II}
\section{Pushforward measure}%
\label{sec:Pushforward measure}

\begin{definition}[Pushforward function]
    \label{def:9.1}
    Let $(\Omega^{ }, \mathcal{F}^{ })$ and $(\Omega^{*}, \mathcal{F}^{*})$ be two measurable
    spaces and $g: \Omega \to \Omega^{*}$ be $\mathcal{F}/\mathcal{F}^{*}$ measurable. Let
    $\mu$ be a measure on $\mathcal{F}$. Define the function
    \[
    \mu g^{-1}(A^{*}) = \mu(g^{-1}(A^{*})) =
    \mu(\{\omega \in  \Omega: g(\omega \in A^{*})\} ), \quad A^{*} \in \mathcal{F}^{*}
    .\] 
    The measure $\mu g^{-1}$ is referred to as the pushforward measure of $\mu$.
    This means that $\mu g^{-1}$ measures, in terms of $\mu$, the pre-image of each set
    $A^{*}$ under $g$. Hence, $\mu$ is a valid measure on $(\Omega^{*}, \mathcal{F}^{*})$!!
    It provides a way to "transfer" the measure from $(\Omega^{ }, \mathcal{F}^{ })$ to
    $(\Omega^{*}, \mathcal{F}^{*})$ via the function $g$.
\end{definition}

\begin{proposition}[]
    \label{prop:9.1}
    TODO
\end{proposition}

\section{Densities}%
\label{sec:Densities}
\begin{proposition}[$\nu$ is a measure on $\mathcal{F}$]
    \label{prop:9.2}
    Let $(\Omega, \mathcal{F}, \mu)$ be a measure space and $\phi: \Omega \to 
    \overline{\mathbb{R}}$ be a nonnegative and $\mathcal{F}$ measurable function. Then,
    $\nu$ defined by
    \[
    \nu(A) = \int_{A} \phi(\omega)\mu(d\omega), \quad A \in \mathcal{F}, 
    \] 
    is a measure on $\mathcal{F}$
\end{proposition}

\begin{definition}[$\phi$, density of $\nu$ in respect to $\mu$]
    \label{def:9.2}
    Let $(\Omega, \mathcal{F}, \mu)$ be a measure space and $\nu$ be a measure on $\mathcal{F}$.
    A nonnegative and $\mathcal{F}$ measurable funtion $\phi: \Omega \to \overline{\mathbb{R}}$ 
    is said to be a density of $\nu$ with respect to $\mu$ if for any $A \in \mathcal{F},
    \nu(A) = \int_{A} \phi(\omega)\mu(d\omega)$.
\end{definition}

\begin{proposition}[]
    \label{prop:9.3}
    Let $(\Omega, \mathcal{F}, \mu)$ be a measure space. Suppose that $\nu$ is a measure on
    $\mathcal{F}$ with density $\phi$ with respect to $\mu$. Then
    \begin{enumerate}[label=(\roman*)]
        \item for any nonnegative and $\mathcal{F}$ measurable function $f$,
            \[
            \int_{A} f(\omega)\nu(d\omega) = \int_{A} f(\omega)\phi(w)\mu(d\omega), \quad
            A \in \mathcal{F};
            \] 
        \item $f$ is integrable with respect to $\nu$ if and only if $f \phi$ (the product
            of the two functions) is integrable with respect to $\mu$. This is clear in (i).
        \item if $f \phi$ is integrable with respect to $\mu$, then (i) holds.
    \end{enumerate}
\end{proposition}

\section{Integration with respect to the Lebesgue measure on the real line}%
\label{sec:Integration with respect to the Lebesgue measure on the real line}

\begin{definition}[]
    \label{def:9.3}
    Consider the measure space $(\mathbb{R}, \mathfrak{B}(\mathbb{R}), \lambda)$, where $\lambda$
    is the Lebesgue measure on the Borel $\sigma$-field $\mathfrak{B}(\mathbb{R})$. In accordance
    with Definition ~\ref{def:8.5}, a $\mathfrak{B}(\mathbb{R})$ measurable function
    $f: \mathbb{R} \to \overline{\mathbb{R}}$ is Lebesgue integrable if
    $\int_{\mathbb{R}} |f(x)| \lambda(dx) < \infty$. The integral of $f$ with respect to $\lambda$
    is denoted with $\int_{\mathbb{R}}f(x)dx$, i.e., $\int_{\mathbb{R}} f(x)dx = 
    \int_{\mathbb{R}} f(x)\lambda(dx)$. If $E \subset \mathbb{R}$ and $\lambda|_E$ is the
    restriction of $\lambda$ to $\mathfrak{B}(E)$ (cf. Definiton ~\ref{def:4.2}), then a 
    $\mathfrak{B}(E)$ measurable function $f: E \to \overline{\mathbb{R}}$ is referred to as 
    Lebesgue integrable if $\int_{E} |f(x)| \lambda|_E (dx) < \infty$. Also in this case we write
    $\int_{E} |f(x)| \lambda|_E (dx) = \int_{E} f(x)dx$.
\end{definition}

In accordance with the fact that the Lebesgue measure of a single point is zero, we adapt the
following definition.

\begin{definition}[]
    \label{def:9.4}
    TODO. Interesting but easy and well known.
\end{definition}

We review the definition of a Riemann integrable function:

\begin{definition}[title]
    \label{def:title}
    
\end{definition}

\section{Change of variable}%
\label{sec:Change of variable}

\section{Integration on product spaces}%
\label{sec:Integration on product spaces}

\begin{definition}[Product $\sigma$-field]
    \label{def:9.9}
    Let $(X, \mathscr{X})$ and $(Y, \mathscr{Y})$ be two measurable spaces. The product
    $\sigma$-field on the cartesian product $(X \times Y)$ is defined by
    \[
    \mathscr{X} \otimes \mathscr{Y} = \sigma(\{A\times B: A \in \mathscr{X}, B \in \mathscr{Y}\} )
    .\] 
    The definition extends to products of higher order. 

    Consider a collection of measure spaces $(X_1, \mathscr{X}_1), \ldots, (X_n, \mathscr{X}_n)$. We define
    \[
    \otimes_{i=1}^{n}\mathscr{X}_i = \mathscr{X}_1 \otimes \ldots \otimes \mathscr{X}_n
    = \sigma(\{A_1\times \ldots\times A_n: A_i \in \mathscr{X}_i, i =1, \ldots, n\} )
    .\] 
    One can also show that the latter product is associative.
\end{definition}
