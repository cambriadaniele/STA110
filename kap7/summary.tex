\chapter{Measurable functions}%
\label{cha:Measurable functions}
\section{The concept of measurable functions}%
\label{sec:The concept of measurable functions}

\begin{definition}[Measurable function]
    \label{def:7.1}
    Let $(\Omega, \mathcal{F})$ and $(\Omega^{\ast}, \mathcal{F}^{\ast})$ be two measurable spaces 
    (cf. Definition ~\ref{def:4.3}). A function $f: \Omega \to \Omega^{\ast}$ is said to be measurable 
    $\mathcal{F}/\mathcal{F}^{\ast}$ if for any $A^{\ast} \in \mathcal{F}^{\ast}, f^{-1}(A^{\ast}) \in \mathcal{F}$.
\end{definition}

\begin{proposition}[Measurable function]
    \label{prop:7.1}
    let $(\Omega, \mathcal{F})$ and $(\Omega^{\ast}, \mathcal{F}^{\ast})$ be two measurable spaces and 
    $f: \Omega \to \Omega^{\ast}$ be a function. Suppose that $\mathcal{F}^{\ast} = \sigma(\mathcal{G})$ 
    and for any $G \in  \mathcal{G}$, $f^{-1}(G) \in \mathcal{F}$. Then, $f$ is 
    $\mathcal{F}/\mathcal{F}^{\ast}$ measurable.
\end{proposition}

\begin{definition}[Borel function]
    \label{def:7.2}
    A function $f: \mathbb{R}^{m} \to \mathbb{R}^{k}$ is called Borel function if it is measurable 
    $\mathfrak{B}(\mathbb{R}^{m})/\mathfrak{B}(\mathbb{R}^{k})$.
\end{definition}

\begin{proposition}[Continuous functions and Borel functions]
    \label{prop:7.2}
    Any continuous function $f: \mathbb{R}^{m} \to \mathbb{R}^{k}$ is a Borel function.
\end{proposition}

\begin{proposition}[$\mathcal{F}/\mathfrak{B}(\mathbb{R})$ measurable]
    \label{prop:7.3}
    Let $(\Omega, \mathcal{F})$ be a measurable space and $f: \Omega \to \mathbb{R}$ be a real-valued function. Suppose
    that $\{\omega \in \Omega: f(\omega) \le x\} \in \mathcal{F}$ for any $x \in  \mathbb{R}$, then $f$ is 
    $\mathcal{F}/\mathfrak{B}(\mathbb{R})$ measurable. In other words: $f$ is a measurable function if the pre-image of 
    any interval $(-\infty, x]$ under $f$ is a measurable set in $\mathcal{F}$, or $f^{-1}((-\infty, x]) \in \mathcal{F}$.
    since $\mathfrak{B}(\mathbb{R}) = \sigma(\{(-\infty, x]: x \in  \mathbb{R}\})$, we also clearly see the proof
    (cf. Proposition ~\ref{prop:7.1}).
\end{proposition}

\paragraph{Thinking about $f^{-1}((-\infty, x))$}%
\label{ta:7.1}
If $B \in  \mathfrak{B}(\mathbb{R})$, then,
$ f^{-1}(B) = \{\omega \in \Omega: f(\omega) \in B\} $
Is the same as saying,
$ f^{-1}((-\infty, x)) = \{\omega \in \Omega: f(\omega) \le x\} $. $f^{-1}(B)$ will return ALL of the values $\omega \in \Omega$ 
for which $f(\omega) \in B$. See My Example ~\ref{myex:7.1} for further intuition.


\paragraph{Define $\mathbbm{1}_A(\omega)$} TODO
\label{}

\begin{example}[Simple measurable function]
    \label{ex:7.3}
    Let $\Omega = \{h, t\}$ and $\mathcal{F} = \mathcal{P}(\{h, t\}) = \{\emptyset, \{h\}, \{t\}, \{h, t\} \} $. Then,
    $\{h\} \in \mathcal{P}(\{h, t\}) $. Thus
    \[
    f(\omega) = 
    \begin{cases}
        1, & \text{if } \omega = h, \\
        0, & \text{if } \omega =t,
    \end{cases}
    \]
    is $\mathcal{P}(\{h, t\})/{\mathfrak{B}(\mathbb{R})}$ measurable.
    In order for f to be $\mathcal{P}(\{h, t\})/{\mathfrak{B}(\mathbb{R})}$ measurable, the pre-image of every
    Borel set in $\mathbb{R}$ under f must be an element of $\mathcal{F}$. For any $x \in \mathbb{R}$,
    $f^{-1}((-\infty, x])$ will either be $\emptyset, \{h\}, \text{or } \{t\} \in \mathcal{F}$. 
\end{example}

\begin{proposition}[$\mathcal{F}/\mathfrak{B}(\mathbb{R}^{k})$ measurable]
    \label{prop:7.4}
    Let $(\Omega, \mathcal{F})$ be a measurable space and $f: \Omega \to \mathbb{R}^{k}$, i.e.,
    \[
    f(\omega) = (f_1(\omega), \ldots, f_k(\omega))
    .\] 
    Then, $f \text{ is } \mathcal{F}/\mathfrak{B}(\mathbb{R}^{k})$ measurable if and only if for any
    $i = 1, \ldots, k, f_i: \Omega \to \mathbb{R} \text{ is } \mathcal{F}/\mathfrak{B}(\mathbb{R})$ measurable.
\end{proposition}

\begin{proposition}[Composite measurable function]
    \label{prop:7.5}
    Let $(\Omega, \mathcal{F})$ be a measurable space and $f_i: \Omega \to \mathbb{R}, i = 1, \ldots, k, \text{ be }
    \mathcal{F}/\mathfrak{B}(\mathbb{R})$ measurable. Suppose that $g: \mathbb{R}^{k} \to \mathbb{R}$ is
    $\mathfrak{B}(\mathbb{R}^{k})/\mathfrak{B}(\mathbb{R})$ measurable. Then,
    \[
    w \mapsto g((f_1(\omega), \ldots, f_k(\omega))) = g(f_1(\omega), \ldots, f_k(\omega))
    .\] 
    is $\mathcal{F}/\mathfrak{B}(\mathbb{R})$ measurable. (Composite function usually written without double brackets)
\end{proposition}

\begin{proposition}[Continuity preserves measurability in function composition]
    \label{prop:7.6}
    Let $(\Omega, \mathcal{F})$ be a measurable space and $f_i: \Omega \to \mathbb{R}, i = 1, \ldots, k, \text{ be }
    \mathcal{F}/\mathfrak{B}(\mathbb{R})$ measurable. Then, if $g: \mathbb{R}^{k} \to \mathbb{R}$ is continuous,
    \[
    w \mapsto g(f_1(\omega), \ldots, f_k(\omega))
    .\] 
    is $\mathcal{F}/\mathfrak{B}(\mathbb{R})$ measurable. 
\end{proposition}

\begin{example}[Continuity preserves measurability]
    \label{ex:7.5}
    Let $(\Omega, \mathcal{F})$ be a measurable space and $f_i: \Omega \to \mathbb{R}, i = 1, \ldots, k, \text{ be }
    \mathcal{F}/\mathfrak{B}(\mathbb{R})$ measurable. Then, $\sum_{i=1}^{k} f_i$ is $\mathcal{F}/\mathfrak{B}(\mathbb{R})$
    measurable (cf. Proposition ~\ref{prop:2.12}).
\end{example}

\begin{example}[Continuity preserves measurability]
   \label{ex:7.6}
    Let $(\Omega, \mathcal{F})$ be a measurable space and $f_i: \Omega \to \mathbb{R}, i = 1, \ldots, k, \text{ be }
    \mathcal{F}/\mathfrak{B}(\mathbb{R})$ measurable. Then, $\prod_{i=1}^{k} f_i $ is $\mathcal{F}/\mathfrak{B}(\mathbb{R})$
    measurable (cf. Proposition ~\ref{prop:2.12}).
\end{example}

\begin{definition}[Simple functions]
    \label{def:7.3}
    A function $f: \Omega \to \mathbb{R}$ is called simple if there exists $n \in \mathbb{N}, \alpha_1, \ldots, \alpha_n
    \in \mathbb{R}$ and sets $A_1, \ldots, A_n \subset \Omega$ s.t.
    \[
        f(\omega) = \sum_{i=1}^{n} \alpha_i \mathbbm{1}_{A_i}(\omega) \quad \omega \in \Omega
    .\] 
    That is, a simple function is a finite linear combination of indicator functions.
\end{definition}

\begin{example}[Simple function]
    \label{ex:7.7}
    Let $(\Omega, \mathcal{F})$ be a measurable space and $f$ be a simple function on $\Omega$, i.e.,
    $f(\omega) = \sum_{i=1}^{n} \alpha_i \mathbbm{1}_{A_i} (\omega)$. Then, if $A_i \in \mathcal{F}$ for any
    $i = 1, \ldots, n, f$ is $\mathcal{F}/\mathfrak{B}(\mathbb{R})$ measurable.
\end{example}

\begin{myexample}[Simple function]
    \label{myex:7.1}
    Let $(\Omega, \mathcal{F})$ be a measurable space and $f: \Omega \to \mathbb{R}$ be the function defined
    in ~\ref{def:7.3}. For this simplified setting, suppose $\Omega = \{1,2,3,4\} \text{ and }\mathcal{F} =
    \{\emptyset, \{1,2\} , \{3,4\} , \Omega\} $. Moreover, we define our function with $n =2$, where
     $\alpha_1 = 3, \alpha_2 = 5, A_1 = \{1,2\} \text{ and } A_2 = \{3,4\} $. Then,
     \[
         f(\omega) = 3\cdot \mathbbm{1}_{\{1,2\}}(\omega) + 5\cdot \mathbbm{1}_{\{3,4\}}(\omega)
     .\] 
     Now, let's consider two preimages of this function, $f^{-1}(\{3\} ) \text{ and } f^{-1}(\{12\} )$. Note that
     both of these sets are Borel sets in $\mathbb{R}$. Also note that, if $B \in  \mathfrak{B}(\mathbb{R})$, then,
     \[
     f^{-1}(B) = \{\omega \in \Omega: f(\omega) \in B\} 
     .\] 
     As seen in Thinking about ~\ref{ta:7.1}. Since $f$ takes the value $3$ for $\omega \in \{1,2\} $,
     $f^{-1}(\{3\} ) = \{1,2\} \in \mathcal{F} $. And, as $f$ doesn't take any value for values
     $\not\in \{\{1,2\} , \{3,4\} \} $, $f^{-1}(\{12\} ) = \emptyset \in \mathcal{F}$. So indeed, $f$ is
     $\mathcal{F}/\mathfrak{B}(\mathbb{R})$ measurable.
\end{myexample}

\begin{definition}[Simple functions in standard form]
    \label{def:7.4}
    Let $(\Omega, \mathcal{F})$ be a measurable space and $f: \Omega \to \mathbb{R}$ be a simple function, as defined in
    Definition ~\ref{def:7.3}. $f$ is called standard if $\cup_{i=1}^{n}A_i = \Omega$ and $\{A_1, \ldots, A_n\} \subset \mathcal{F}$
    is disjoint. if $f$ is standard, we say that it is a simple function in standard form.
\end{definition}

\begin{proposition}[7.7]
    \label{prop:7.7}
    TODO
    
\end{proposition}

\begin{proposition}[7.8]
    \label{prop:7.8}
    TODO
    
\end{proposition}

\section{Functions taking values in the extended real numbers}%
\label{sec:Functions taking values in the extended real numbers}
\begin{definition}[Measurable functions in $\overline{\mathbb{R}}$]
    \label{def:7.5}
    Let $(\Omega, \mathcal{F})$ be a measurable space and $f: \Omega \to \overline{\mathbb{R}}$. We say that
    $f$ is $\mathcal{F}$ measurable if for any $A \in \mathfrak{B}(\mathbb{R})$, $\{\omega \in  \Omega: f(\omega) \in A\} \in \mathcal{F}$ 
    and $\{\omega \in  \Omega: f(\omega) = -\infty\} \in \mathcal{F}$ and $\{\omega \in  \Omega: f(\omega) = \infty\} \in \mathcal{F}$ .
    Or, in other words, $f^{-1}(A), f^{-1}(-\infty), f^{-1}(\infty) \in \mathcal{F}$.
\end{definition}

\paragraph{Remark 7.2}
\label{rem:7.2}
As seen in the script, as, if $f: \Omega \to \mathbb{R}, f^{-1}(-\infty), f^{-1}(\infty) = \emptyset$, any results on $\mathcal{F}$ meeasurable
functions $f: \Omega \to \overline{\mathbb{R}}$ also apply to $\mathcal{F}/\mathfrak{B}(\mathbb{R})$ measurable functions $f:\Omega\to \mathbb{R}$.

\paragraph{Remark 7.3}
\label{rem:7.3}
TODO, but important for notation, read it from the script.

\begin{proposition}[7.9]
    \label{prop:7.9}
    TODO
\end{proposition}

\begin{proposition}[7.10]
    \label{prop:7.10}
    TODO
\end{proposition}

\begin{definition}[Positive and negative parts of a function]
    \label{def:7.6}
    TODO
\end{definition}

\begin{proposition}[]
    \label{prop:7.11}
    This proposition states that any $\mathcal{F}$-measurable function $f$ can be approximated by a sequence of $\mathcal{F}$-measurable simple functions $(f_n)_{n \in \mathbb{N}}$ such that $f_n(\omega) \rightarrow f(\omega)$ for all $\omega \in \Omega$.

\end{proposition}

\begin{myexample}[]
    \label{myex:prop7.11}
    Consider $\Omega = [0,1]$ and $\mathcal{F}$ be the Borel $\sigma$-field on $[0,1]$. Let $f(x) = x$. Define the sequence of simple functions $f_n(x) = \frac{\lfloor nx \rfloor}{n}$. Each $f_n$ is $\mathcal{F}$-measurable and $f_n(x) \rightarrow x$ as $n \rightarrow \infty$.
\end{myexample}

\begin{proposition}[]
    \label{prop:7.12}
This proposition extends 7.11 by specifying that if $f$ is non-negative, the convergence of the simple functions can be made monotone, i.e., $f_n(\omega)$ increases with $n$ and converges to $f(\omega)$.

\end{proposition}

\begin{myexample}[]
    \label{myex:prop7.12}
Using the same function $f(x) = x$ on $\Omega = [0,1]$, define $f_n(x) = \frac{\lfloor nx \rfloor}{n}$. Note that $f_n(x) \leq f_{n+1}(x)$ for all $x \in [0,1]$ and $n \in \mathbb{N}$, ensuring that $f_n(x) \uparrow f(x)$ as $n \rightarrow \infty$.
\end{myexample}

\section{Sequence of measurable functions}%
\label{sec:Sequence of measurable functions}
