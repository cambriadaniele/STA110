\chapter{Measurable sets: Part I}%
\label{cha:Measurable sets: Part I}
\section{Measurable spaces}%
\label{sec:Measurable spaces}

\begin{definition}[$\sigma$-field]
    \label{def:4.1}
    Let $\Omega$ be a nonempty set. A family of subsets $\mathcal{F}$ of $\Omega$ is called
    a $\sigma$-field on $\Omega$ if the following three itmes are statisfied:
    \begin{enumerate}[label=(\roman*)]
        \item $\Omega \in \mathcal{F}$;
        \item $A \in \mathcal{F} \Rightarrow A^{c} \in \mathcal{F}$;
        \item if $\{ A_i: i \in  \mathbb{N}\} $ is a collection of sets s.t. $A_i \in \mathcal{F}$ 
            for any $i \in \mathbb{N}$, then $\bigcup_{i \in \mathbb{N}} A_i \in \mathcal{F}$.
    \end{enumerate}
\end{definition}

\begin{definition}[]
    \label{def:4.2}
    4.2 TODO
\end{definition}

\begin{definition}[Measurable space]
    \label{def:4.3}
    let $\Omega \neq \emptyset$ and $\mathcal{F}$ be a $\sigma$-field on $\Omega$. The pair $(\Omega, \mathcal{F}$ is referred 
    to as a measurable space. if $A \in \mathcal{F}$, then A is said to be measurable. if $A\subset \mathcal{F}$ and
    $\mathcal{A}$ is a $\sigma$-field on $\Omega$, $\mathcal{A}$ is referred to as a sub-$\sigma$-field on $\Omega$.
\end{definition}
